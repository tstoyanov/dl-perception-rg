%%
\documentclass[savetrees,12pt]{article}

\usepackage[pdftex]{graphicx}
\usepackage{amsmath}
%\usepackage{amsthm}
\usepackage{amsfonts}
\usepackage{amssymb}
\usepackage{subfigure}
\usepackage{natbib}
\usepackage{url}
\usepackage[lined,algonl,ruled]{algorithm2e}
\usepackage{geometry}
%\usepackage{algorithm}
\usepackage{listings}

\usepackage{pdfpages}
\usepackage[utf8]{inputenc}

\geometry{
    body={6.5in, 8.5in},
	left=1.0in,
	top=1.2in
}

\newcommand{\mbm}[1]{\mbox{\boldmath $#1$}}


\begin{document}

\title{Practice Autoencoders}
\author{50DT057}
\date{Due on 15.03.2018}
%% use optional labels to link authors explicitly to addresses:
%% \author[label1,label2]{<author name>}
%% \address[label1]{<address>}
%% \address[label2]{<address>}
\maketitle

\section{Learning to reconstruct images}
In this example, we want to experiment with autoencoders for reconstructing images. We will use as a base code from this github project: \url{https://github.com/aymericdamien/TensorFlow-Examples}.
The code has already been imported in our local repository and is located under \verb+dl-perception-rg/sample_code/chapter_14+.
Modify the file \verb+autoencoder_original.py+ to:

\begin{itemize}
\item Use the TensorFlow estimator API, store/restore checkpoints, and display the loss information in TensorBoard.
\item Add one more layer in the encoder / decoder architecture and observe the effect on the final loss. 
\item Compare the two-layer and three-layer autoencoder with the same capacity.
\end{itemize}


\end{document}

